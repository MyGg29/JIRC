%%
%% Authors: BECART Quentin - GOURMELON Gwendal
%%

% Preamble
\documentclass[11pt]{article}



% Packages
\usepackage{amsmath}
\usepackage[T1]{fontenc}
\usepackage[utf8]{inputenc}
\usepackage{natbib}
\usepackage{graphicx}
\usepackage{hyperref}
\usepackage{float}
\usepackage{here}
\begin{document}

    % Titre
    \title{JMessenger - Rapport de projet Java}
    \author{BECART Quentin - GOURMELON Gwendal}
    \date{ISEN AP4 - Mars 2019}

    % Document

    \maketitle

    \section{Introduction}
    Nous avons au cours de cette période école et dans le cadre du module de programmation Java, été amenés à réaliser un programme de messagerie instantanée.\newline
    \centerline{https://github.com/MyGg29/JIRC}
    \section{Rappel du cahier des charges}
    \begin{itemize}
        \item Le programme devra permettre à tous les utilisateurs d’envoyer et de recevoir des messages sur un salon ou à des utilisateurs spécifiqués identifiés (1 ou plusieurs).
        \item Le programme devra pouvoir permettre de créer/modifier des salons et d’y inviter des utilisateurs
        \item Chaque utilisateur possèdera ou non les droits de modification d’un salon. Droits qu’il pourra répliquer ou non aux autres utilisateurs du salon.
        \item Le programme doit disposer d’une interface graphique claire pour la lecture de l’historique des salons, la liste des discussions en cours, les utilisateurs connectés sur chaque salon.
        \item Le programme devra permettre, pour les administrateurs d’un salon, d’exporter l’historique de discussion d’un salon via un fichier XML.
        \item Chaque utilisateur peut, à tout moment, se désinscrire d’un salon.
        \item La déconnexion du salon ne doit pas entraîner la perte de l’historique des conversations.
        \item Le programme devra fournir des statistiques sur l’activité des utilisateurs et des différents salons.
    \end{itemize}

    \section{Structure du programme}
    \subsection{MVC}
    Dans le cadre de ce projet, nous avons utilisé l'architecture logicielle MVC (pour Modèle-Vue-Contrôleur) se présentant de la manière suivante :
    \begin{itemize}
        \item Le Modèle contient l'ensemble des données à afficher
        \item La Vue constitue l'interface graphique (elle est l'élément d'interaction avec l'utilisateur)
        \item Le Contrôleur représente le lien entre la vue et le modèle et définit les actions qui seront effectuées lors d'un évènement (interaction de l'utilisateur)
    \end{itemize}
    Le modèle et le contrôleur sont tous deux des fichiers .java, la vue est quant à elle un fichier .fxml.

    \subsection{Packages}
    Nous avons actuellement 4 packages :

    \begin{itemize}
        \item Client : côté client
        \item Database : partie base de données
        \item Models : modèles, autrement dit les classes génériques d’informations utilisées
        \item Server : côté serveur
        \item Util : partie utilitaire (fonctions génériques, ...)
    \end{itemize}


    \subsection{Classes importantes}

    \begin{itemize}
        \item client.Main.java : le point d'entrée du Client.
        \item serveur.Server.java : le point d'entrée du serveur. Créer un Thread pour chaque nouvelle connexion
        \item client.controllers.MainController.java : le contrôleur de la fenêtre principale de chat
        \item models.Client.java : La logique client. S'occupe du transfert de données entre le client et le serveur
        \item models.ClientListener.java : Thread d'écoute du client. Permettant de parler et d'écouter le serveur en même temps.
        \item serveur.SSocket.java : La logique du serveur.
    \end{itemize}


    \subsection{Diagramme de classes et diagramme client}

    Voir pages suivantes.
    L'image du diagramme représenté est disponible en meilleure résolution sur le repository GitHub du projet.\citep{jmessengerbecartgourmelon}

    \begin{figure}[!h]
        \centering
        \centerline{\includegraphics[scale=0.25]{Diagramme_Client.png}}
        \caption{Diagramme de classes}
        \label{fig:diagClasses}
    \end{figure}

    \begin{figure}[!h]
        \centering
        \centerline{\includegraphics[scale=0.30]{Diagramme_de_classe.png}}
        \caption{Diagramme Client}
        \label{fig:diagClient}
    \end{figure}

    \section{Conclusion}

    Ce projet nous a permis de mettre en pratique les différents points du langage Java abordés en cours.
    Nous nous sommes également familiarisés avec le framework FXML et le développement d'interfaces graphiques d'une manière générale. Il aura aussi été interessant de développer deux applications qui communiquent entre elles : le serveur d'un côté et le client de l'autre.



    \bibliographystyle{plain}
    \bibliography{references}

\end{document}

