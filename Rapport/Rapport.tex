%%
%% Authors: BECART Quentin - GOURMELON Gwendal
%%

% Preamble
\documentclass[11pt]{article}
\documentclass{article}


% Packages
\usepackage{amsmath}
\usepackage[T1]{fontenc}
\usepackage[utf8]{inputenc}

% Titre
\title{JMessenger - Rapport de projet Java}
\authors{BECART Quentin - GOURMELON Gwendal}
\date{ISEN AP4 - Mars 2019}

% Document
\begin{document}

\maketitle

\section{Introduction}
    Nous avons au cours de cette période école et dans le cadre du module de programmation Java, été amenés à réaliser
    un programme de messagerie instantanée.

\section{Rappel du cahier des charges}
\begin{itemize}
    \item Le programme devra permettre à tous les utilisateurs d’envoyer et de recevoir des messages sur un salon ou à des utilisateurs spécifiqués identifiés (1 ou plusieurs).
    \item Le programme devra pouvoir permettre de créer/modifier des salons et d’y inviter des utilisateurs
    \item Chaque utilisateur possèdera ou non les droits de modification d’un salon. Droits qu’il pourra répliquer ou non aux autres utilisateurs du salon.
    \item Le programme doit disposer d’une interface graphique claire pour la lecture de l’historique des salons, la liste des discussions en cours, les utilisateurs connectés sur chaque salon.
    \item Le programme devra permettre, pour les administrateurs d’un salon, d’exporter l’historique de discussion d’un salon via un fichier XML.
    \item Chaque utilisateur peut, à tout moment, se désinscrire d’un salon.
    \item La déconnexion du salon ne doit pas entraîner la perte de l’historique des conversations.
    \item Le programme devra fournir des statistiques sur l’activité des utilisateurs et des différents salons.
\end{itemize}

\section{Structure du programme}
\subsubsection{MVC}
    Dans le cadre de ce projet, nous avons utilisé l'architecture logicielle MVC (pour Modèle-Vue-Contrôleur) se présentant de la manière suivante :
\begin{itemize}
    \item Le Modèle contient l'ensemble des données à afficher
    \item La Vue constitue l'interface graphique (elle est l'élément d'interaction avec l'utilisateur)
    \item Le Contrôleur représente le lien entre la vue et le modèle et définit les actions qui seront effectuées lors d'un évènement (interaction de l'utilisateur)
\end{itemize}
    Le modèle est le contrôleur sont tous deux des fichiers .java, la vue est quant à elle un fichier .fxml.

\subsubsection{Packages}
    Nous avons actuellement 4 packages :

\begin{itemize}
    \item Client : côté client
    \item Database : partie base de données
    \item Models : modèles, autrement dit les classes génériques d’informations utilisées
    \item Server : côté serveur
\end{itemize}

\subsubsection{Diagramme de classes / Arborescence globale}
    Dans cette partie, nous allons aborder la manière dont sont réparti les classes et ainsi visualiser une arborescence globale du projet.
    Pour notre illustration, notons que l’arborescence se divise, de manière générale, de la manière suivante :

\begin{itemize}
    \item PACKAGE
        \subitem SOUS PACKAGE (éventuel)
            \subsubitem CLASSES
\end{itemize}


\begin{itemize}
    \item Client
        \subitem Controllers (.java)
            \subsubitem ConnexionController
            \subsubitem JoinChannelController
            \subsubitem MainController
            \subsubitem ParametresController
        \subitem CSS (.css)
            \subsubitem common
        \subitem Ressources
            \subsubitem (images)
        \subitem Views (.fxml)
            \subsubitem ChannelParameters
            \subsubitem Client
            \subsubitem Connexion
            \subsubitem JoinChannel
            \subsubitem Stats
        \subitem Database (.java)
            \subsubitem DatabaseClass
        \subitem Models (.java)
            \subsubitem Client
            \subsubitem Function
            \subsubitem Message
            \subsubitem TypesChannel
            \subsubitem TypesMessage
        \subitem Server (.java)
            \subsubitem Channel
            \subsubitem Server (Server = SSocket)
            \subsubitem User
\end{itemize}


\end{document}

